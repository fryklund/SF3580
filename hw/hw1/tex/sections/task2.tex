%!TeX spellcheck = en-US
%!TEX root = ../hw1_report.tex

\subsection{$(a)$}
Insert figure
\subsection{$(b)$}
Insert figure
\subsection{$(c)$}
Insert figure

The Rayleigh quotient only uses the symmetric part of $A$ in
\begin{equation*}
  r(\mathbf{x}) = \mathbf{x}^{H}A\mathbf{x}
\end{equation*}
assuming $\mathbf{x}$ is normalized.
The matrix $A$ is no longer symmetric, i.e. $A\neq A^{H}$, but any square matrix can be decomposed into a symmetric part $A_{s}$ and a nonsymmetric part $A_{ns}$ by
\begin{equation*}
  A = \underbrace{\frac{1}{2}\left(A + A^{H}\right)}_{= A_{s}} + \underbrace{\frac{1}{2}\left(A - A^{H}\right)}_{=A_{ns}}.
\end{equation*}
Thus
\begin{equation*}
  r(\mathbf{x}) = \mathbf{x}^{H}A_{s}\mathbf{x} + \mathbf{x}^{H}A_{ns}\mathbf{x} = \mathbf{x}^{H}A_{s}\mathbf{x}
\end{equation*}
since
\begin{equation*}
 \mathbf{x}^{H}A_{ns}\mathbf{x}=\mathbf{x}^{H}A\mathbf{x} - \mathbf{x}^{H}A^{H}\mathbf{x} = 0.
\end{equation*}
For a nonsymmetric matrix all avaliable information is not used.
