%!TeX spellcheck = en-US
%!TEX root = ../hw3_report.tex
\subsection*{(a)}
%We start by observing that if $\lambda$ and $v$ are an eigenpair for $A$, then $t\lambda$ is an eigenvalue of $tA$ since $tAv = t\lambda v$.
%
%Let $\mu \in \mathbb{C}$ be an expansion point, then
%\begin{equation}
%  f(tA) = \sum\limits_{i = 0}^{\infty} \frac{f^{(i)}(\mu)}{i!}(tA-\mu I)^{i}.
%\end{equation}
%If $A\in\mathbb{C}^{n\times n}$, then $f:C^{n \times n}\rightarrow C^{n \times n}$. Let $f(z) = \exp (z)$ and $g(z) = dz(t)/dt$, then
%\begin{equation}
%  g(f(tA)) = g\left(\sum\limits_{i = 0}^{\infty} \frac{\exp(\mu)}{i!}(tA-\mu I)^{i}\right) = \sum\limits_{i = 0}^{\infty}\left( \frac{\exp(\mu)}{i!}g((tA-\mu I)^{i})\right) = A\sum\limits_{i = 1}^{\infty}\frac{\exp(\mu)}{(i-1)!}\left(tA-\mu I\right)^{i-1},
%\end{equation}
%be rearranging the indices since $(tA-\mu I)^{0} = I$. The above holds if $g((tA)^{i}) = iA(tA)^{i-1}$. How to show this?

%Anna's attempt:
Consider the function $f(z,t) = e^{tz}$. We want to investigate the matrix valued function $f(A,t) = e^{Az}$. Let $\mu \in \mathbb{C}$ be an expansion point. Then,
\begin{equation}
  f(A,t) = \sum\limits_{i = 0}^{\infty} \frac{f^{(i)}(\mu)}{i!}(A-\mu I)^{i} = \sum\limits_{i = 0}^{\infty} \frac{t^ie^{t\mu}}{i!}(A-\mu I)^{i}.
\end{equation}
If $A\in\mathbb{C}^{n\times n}$, then $f:C^{n \times n}\rightarrow C^{n \times n}$. Now, compute the derivative of $f(A,t)$ with respect to time:
\begin{equation}
\begin{aligned}
\frac{\mathrm d}{\mathrm d t}e^{tA} & = \frac{\mathrm d}{\mathrm dt}\sum^{\infty}_{i = 0} \frac{t^ie^{t\mu}}{i!}(A-\mu I )^i = \\
& = \sum^{\infty}_{i = 0} \frac{\mathrm d}{\mathrm dt}\left(\frac{t^ie^{t\mu}}{i!}(A-\mu I )^i\right) = \\
& = \sum^{\infty}_{i = 0} \frac{\mathrm d}{\mathrm dt}\left(\frac{t^ie^{t\mu}}{i!}\right)(A-\mu I )^i  = \\
&= \sum^{\infty}_{i = 0}  \left(\frac{it^{i-1}e^{t\mu}+t^i\mu e^{t\mu}}{i!}\right) (A-\mu I )^i.
\end{aligned}
\end{equation}
The last expression can be identified as $g(A)$, where $g(z) = ze^{tz}$ as  the expression
\begin{equation}
\left(it^{i-1}e^{t\mu}+t^i\mu e^{t\mu}\right)
\end{equation}
is the $i$th derivative of the product $z\cdot e^{tz}$, which can be seen using the general Leibniz rule ($(f_1f_2)^{(n)} = \sum_{k = 0}^n\binom{n}{k}f_1^{(n-k)}(x)f_2^{(k)}(x)$). Thus, we can conclude that $\frac{\mathrm d }{\mathrm dt}e^{tA} = Ae^{tA}$. The matrix function $e^{tA}A$ has the same Taylor expansion expression as $Ae^{tA}$. Thus, $\frac{\mathrm d }{\mathrm dt}e^{tA} = Ae^{tA} = e^{tA} A$, which is what we wanted to show.




%Let $f(z) = \exp (z)$ and $g(z) = dz(t)/dt$, then
%\begin{equation}
%  g(f(tA)) = g\left(\sum\limits_{i = 0}^{\infty} \frac{\exp(\mu)}{i!}(tA-\mu I)^{i}\right) = \sum\limits_{i = 0}^{\infty}\left( \frac{\exp(\mu)}{i!}g((tA-\mu I)^{i})\right) = A\sum\limits_{i = 1}^{\infty}\frac{\exp(\mu)}{(i-1)!}\left(tA-\mu I\right)^{i-1},
%\end{equation}
%be rearranging the indices since $(tA-\mu I)^{0} = I$. The above holds if $g((tA)^{i}) = iA(tA)^{i-1}$. How to show this?
\subsection*{(b)}
Introduce
\begin{equation}
[B,A]_{n} =  [[B,A]_{n-1},A], n = 0,1,2,\ldots, \quad \text{ where }[B,A]_{1} = [B,A] = BA-AB \text{ and } [B,A]_{0} = B,
\end{equation}
which satisfies $[A+B,C]_{n} = [A,C]_{n} + [B,C]_{n}$. This is shown by induction: the initial case is $[A+B,C]_{1} = AC-CA + BC-CB = [A,C]_{1}+[B,C]_{1}$. Now assume $[A+B,C]_{n} = [A,C]_{n} + [B,C]_{n}$ holds, then
\begin{align}
[A+B,C]_{n+1} &=  [AC-CA + BC-CB,A]_{n} = [AC-CA,C]_{n} + [BC-CB,C]_{n} \\
&= [[A,C],C]_{n} + [[B,C],C]_{n}=[A,C]_{n+1} + [B,C]_{n+1}.
\end{align}


Let $G(t) = \exp(-tA)B\exp(tA)$, which is analytic in $t$. Thus we may write
\begin{equation}
  \label{eq:task7bTaylor}
G(t) = \sum\limits_{i = 0}^{\infty}\frac{t^{i}}{i!} G^{(i)}(\mu)=\sum\limits_{i = 0}^{\infty}\frac{t^{i}}{i!} G_{i},
\end{equation}
where $G_{0} = B$.
By (a) we have that
\begin{equation}
  \frac{d}{dt}G(t) = G(t)A-AG(T) = [G(t),A].
\end{equation}
Setting this to be equal to the the derivative of \eqref{eq:task7bTaylor} with respect to $t$ gives
\begin{equation}
\sum\limits_{i = 0}^{\infty}\frac{t^{i}}{i!} [G_{i},A] = \sum\limits_{i = 1}^{\infty}\frac{t^{i-1}}{(i-1)!} G_{i}.
\end{equation}
By shifting the indexing from $i = 1,2,\ldots $ to $i = 0,1,\ldots$ for the right hand side we get
\begin{equation}
\sum\limits_{i = 0}^{\infty}\frac{t^{i}}{i!} [G_{i},A] = \sum\limits_{i = 0}^{\infty}\frac{t^{i}}{i!} G_{i+1}.
\end{equation}
We conclude that $G_{i+1} = [G_{i},A]_{i}$, that is $G_{1} = [G_{0},A]_{0} = B$ and
\begin{equation}
  G(t) = B + t[B,A]+\frac{t^{2}}{2!}[[B,A],A]+\frac{t^{2}}{2!}[[B,A],A,A]+\ldots
\end{equation}

\subsection*{(c)}
Det här verkar misstänksamt simpelt...

We identify the integrand as $G(t)$, that is

\begin{equation}
  P = \int\limits_{0}^{\tau}\,\exp(tA^{T})B\exp(tA)\,dt =\int\limits_{0}^{\tau}\,\exp(-tA)B\exp(tA)\,dt = \int\limits_{0}^{\tau}\,G(t)\,dt.
\end{equation}
Introduce
\begin{equation}
  P_{n} = \int\limits_{0}^{\tau}\,\sum\limits_{i=0}^{n}\frac{t^{i}}{i!}G_{i+1}\, dt=\,\sum\limits_{i=0}^{n}\int\limits_{0}^{\tau}\frac{t^{i}}{i!}G_{i+1}\, dt,
\end{equation}
Since the integrand is uniformly convergent, assuming $\tau$ finite, it holds that
\begin{equation}
  \lim\limits_{n\rightarrow \infty}P_{n}=  \lim\limits_{n\rightarrow \infty}\int\limits_{0}^{\tau}\, \sum\limits_{i=0}^{n}\frac{t^{i}}{i!}G_{i+1}\, dt = \int\limits_{0}^{\tau}\,\lim\limits_{n\rightarrow \infty} \sum\limits_{i=0}^{n}\frac{t^{i}}{i!}G_{i+1}\, dt = \int\limits_{0}^{\tau}\,G(t)\,dt = P
\end{equation}
where the limit was moved inside due to the dominated convergence theorem. Furthermore,
\begin{equation}
  \int\limits_{0}^{\tau}\frac{t^{i}}{i!}G_{i+1}\, dt = [G_{i},A]\frac{\tau^{i+1}}{(i+1)!},
\end{equation}
for every $i$. Thus
\begin{equation}
  P  =  \lim\limits_{n\rightarrow \infty} \sum\limits_{i=0}^{n} \int\limits_{0}^{\tau}\,\frac{t^{i}}{i!}G_{i+1}\, dt =  \sum\limits_{i=0}^{\infty}[G_{i},A]\frac{\tau^{i+1}}{(i+1)!}.
\end{equation}


\subsection*{(d)}
Task: Let $C_k = [C_{k-1},A]$, with $C_0 = B$. We want to show that $\|C_k\|\leq 2^k\|A\|^k\|B\|$.

The proof is done by induction. For $k = 0$ we have that $\|C_0\| = \|B\|\leq 2^0\|A\|^0\|B\|$. Now, assume that $\|C_k\|\leq 2^k\|A\|^k\|B\|$. We want to show that
$\|C_{k+1}\|\leq 2^{k+1}\|A\|^{k+1}\|B\|$:

\begin{equation}
\begin{aligned}
\|C_{k+1}\| = \|C_kA-AC_k\| = \|C_{k}A+(-AC_k)\|\leq\|C_kA\|+\|-AC_k\| = \|C_kA+|-1|\|AC_k\|\leq\|C_k\|\|A\|+\|A\|\|C_k\| =\\
= 2\|A\|\|C_k\| = 2^{k+1}\|A\|^{k+1}\|B\|,
\end{aligned}
\end{equation}
which is what we wanted to show.
\subsection*{(e)}
Suppose $\|A\|<\frac{1}{2}$ and $t\leq 1$. Let $G_N(t)$ be the truncation of $G(t)$, where
\begin{equation}
G(t) = \sum^{\infty}_{k = 0}\frac{t^k}{k!}C_k.
\end{equation}
Then,
\begin{equation}
\begin{aligned}
\|G_N(t)-G(t)\|&= \|\sum^{\infty}_{k = N+1}\frac{t^k}{k!}C_k\|\leq\sum^{\infty}_{k = N+1}\left(\frac{t^k}{k!}\right)\|C_k\|\leq\\
&\leq \sum^{\infty}_{k = N+1}\left(\frac{t^k}{k!}\right)2^k\|A\|^k\|B\|\leq \sum^{\infty}_{k = N+1}\left(\frac{t^k}{k!}\right)2^k\left(\frac{1}{2}\right)^2\|B\|\leq\frac{\|B\|}{(N+1)!}\sum^{\infty}_{k = N+1}t^k = \frac{\|B\|}{(N+1)!}\sum^{\infty}_{k = 0}t^{k+(N+1)} =\\&=\frac{\|B\|t^{N-1}}{(N+1)!}t^{N+1}\cdot\frac{1}{1-t}.
\end{aligned}
\end{equation}
