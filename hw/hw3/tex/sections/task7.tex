%!TeX spellcheck = en-US
%!TEX root = ../hw3_report.tex
\subsection*{(a)}
We start by observing that if $\lambda$ and $v$ are an eigenpair for $A$, then $t\lambda$ is an eigenvalue of $tA$ since $tAv = t\lambda v$.

Let $\mu \in \mathbb{C}$ be an expansion point, then
\begin{equation}
  f(tA) = \sum\limits_{i = 0}^{\infty} \frac{f^{(i)}(\mu)}{i!}(tA-\mu I)^{i}.
\end{equation}
If $A\in\mathbb{C}^{n\times n}$, then $f:C^{n \times n}\rightarrow C^{n \times n}$. Let $f(z) = \exp (z)$ and $g(z) = dz(t)/dt$, then
\begin{equation}
  g(f(tA)) = g\left(\sum\limits_{i = 0}^{\infty} \frac{\exp(\mu)}{i!}(tA-\mu I)^{i}\right) = \sum\limits_{i = 0}^{\infty}\left( \frac{\exp(\mu)}{i!}g((tA-\mu I)^{i})\right) = A\sum\limits_{i = 1}^{\infty}\frac{\exp(\mu)}{(i-1)!}\left(tA-\mu I\right)^{i-1},
\end{equation}
be rearranging the indices since $(tA-\mu I)^{0} = I$. The above holds if $g((tA)^{i}) = iA(tA)^{i-1}$. How to show this? 
