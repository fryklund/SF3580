%!TeX spellcheck = en-US
%!TEX root = ../hw3_report.tex
Consider the PDE 
\begin{equation}
\begin{aligned}
\Delta u+g(x,y)u &= f(x,y),\text{ for }x,y\in\Omega,\\
u(x,y) & = 0,\text{ for }x,y\in\partial\Omega,
\end{aligned}
\end{equation}
where $\Omega$ is the unit square. 
\subsection*{(a)}
\emph{Derive the 2nd order finite-difference discretization for the grid $x_i = hi, i = 1,\dots,m,y_j = hj,j = 1,\dots,m$ and $h = 1/(m+1)$. Also, derive matrices such that the discretizatioin can be expressed as 
\begin{equation}\label{lyapPDE}
D_{xx}U+UD_{xx}+G\circ U = F, 
\end{equation}
where $U_{i,j} \approx u(x_i,y_j)$. }

Using a 2nd order approximation of the second derivatives 
\begin{equation}
\frac{\partial^2 f}{\partial x^2} = \frac{f(x+h)-2f(x)+f(x-h)}{h^2}
\end{equation}
and similarly for $\frac{\partial^2 f}{\partial y^2}$, we can approximate the laplacian by the five point stencil 
\begin{equation}
\Delta u_{i,j} \approx \frac{u_{i-1,j}-4u_{i,j}+u_{i+1,j}+u_{i,j-1}+u_{i,j+1}}{h^2},
\end{equation}
for $i = 1,2,\dots,m$ and $j = 1,2,\dots,m$.
From the boundary condition, we have
\begin{equation}
u_{0,j} = u_{m+1,j}=u_{i,0} = u_{m+1,0} = 0.
\end{equation}  
The structure for the Laplace operator alone is thus the $m^2\times m^2$ matrix
\begin{equation}\label{tridiag}
\begin{bmatrix}
T &I &0 &\dots& 0\\
I & T& I & \dots \\
\vdots &\vdots& \vdots &\vdots& \vdots\\
0 & 0 & 0 &I& T
\end{bmatrix}
\end{equation}
with $T$ being the $m\times m$ matrix with $-4$ on the diagonal, 1 on the sub- and super-diagonals and zero otherwise. 
The entire system for the PDE can be written on the form $Au = b$, where $A$ has the same structure as the matrix in \eqref{tridiag}, but where the diagonal is given by matrices $B_1,B_2,\dots, B_m$, where $B_j = T+h^2\cdot GG_j$, with $GG_j$ having values only on the main diagonal taking the values $g(x_i,y_j)$, $i=1,\dots,m$. The right hand side vector $b$ is given by 
\begin{equation}
b = h^2
\begin{bmatrix}
f(x_i,y_1)\\
f(x_i,y_2)\\
\vdots\\
f(x_i,y_m)
\end{bmatrix}.
\end{equation}


Observe that this is the vecorized form of the equation \eqref{lyapPDE}, where
$D_{xx}$ is a $m\times m$ matrix with -2 on the main diagonal and -1 on the sub- and superdiagonals, $G = g(x_i,y_j)$ and $F = f(x_i,y_j)$. 
\subsection*{(b)}
\emph{Derive explicit expressions for the eigenvalues of $I\otimes D_{xx}+D_{xx}\otimes I$ in the limit $m\to\infty$ and show that the matrix $I\otimes D_{xx}+D_{xx}\otimes I$ is non-singular in the limit.}

First, we identify that the matrix $I\otimes D_{xx}+D_{xx}\otimes I$ is the finite difference approximation of the laplace operator. In the limit $m\to\infty$, the approximation approaches the continuous laplace operator. Therefore, we seek eigenvalues
\begin{equation}
\Delta u = \lambda u.
\end{equation} 
Writing $u$ as $u(x,y) = X(x)Y(y)$, we get the equation
\begin{equation}
X''+Y'' = \lambda XY \Leftrightarrow \frac{X''}{X}+\frac{Y''}{Y} = \lambda,
\end{equation}
enforcing that for some constants $\alpha$ and $\beta$, 
\begin{equation}
X''=\alpha X\text{ and }Y'' = \beta Y.
\end{equation}
Solving these equations along with the homogeneous boundary conditions, we obtain that 
\begin{equation}
\lambda = (k\pi)^2+(l\pi)^2,\quad k,l = 1,2,3,...
\end{equation}
The equations for $X(x)$ and $Y(y)$ has no nontrivial solutions for $\lambda =0$ and thus it can be concluded the matrix is non-singular. 
\subsection*{(c)}
\emph{Let 
\begin{equation}
g(x,y) = \alpha\sqrt{(x-\frac{1}{2})^2+(y-\frac{1}{2})^2}
\end{equation}
and $f(x,y)= |x-y|$.}
\begin{enumerate}
\item Let $\alpha = 0$ and solve the sparse linear system $Au = b$ using $\backslash$. 
\item We compare the above with solving the equation using the matlab command \texttt{lyap}.
\item Let $\alpha = 1$ and solve with $\backslash$
\item Again, let $\alpha = 1$, but use instead \texttt{gmres} to solve the system.
\item Using $\alpha = 1$ and \texttt{gmres}, use \texttt{lyap} as a left pre-conditioner. 
\item Using $\alpha = 0.1$ and \texttt{gmres}, use \texttt{lyap} as a left pre-conditioner. 
\end{enumerate}

\subsection*{(d)}
\emph{Explain the performance in (c) of the the preconditioned gmres. 
}
Consider $A-E$ to be the matrix $A$ obtained by choosing $\alpha = 0$. Then, from the hint in the lecture notes, 
\begin{equation}
(A-E)^{-1} = A^{-1}-A^{-1}EA^{-1}+\mathcal O(\|E\|^2),
\end{equation}
for sufficiently small $\|E\|$. The matrix $E$ is diagonal, and therefore, $\|E\|_2 = \max(g(x_i,y_2))<\alpha \sqrt{2}/2$. Thus, using the solution from the lyap command (with $\alpha = 0$) as a pre-conditioner, the preconditioner is very close to being the inverse of the matrix $A$ where e.g. $\alpha = 0.1$. The consequence is clearly that gmres converges in few iterations. 

\subsection*{(e)}
\emph{Suppose all elements of the matrix $G$ are zero except $G_{m/4,m/2} = 1/h$, where $m\in 4\mathbb Z$. Solve the equation efficiently by using the \texttt{lyap} command.}

We consider the problem as a rank-one modification and use the Sherman-Morrison-Woodbury formula, stating that if $C$ is an invertible square matrix and $u$, $v$ are column vectors of the same size, then $C+uv^T$ is invertible iff $1+v^TC^{-1}u\neq 0$. The inverse is given by
\begin{equation}
(C+uv^T)^{-1} = C^{-1}-\frac{C^{-1}uv^TC^{-1}}{1+v^TC^{-1}u}.
\end{equation}
Using $u = e_{m/4}$ and $v  = (1/h)e_{m/2}$, we can then compute 

A naive approach representing $G$ as a matrix to form the matrix $A$ described in task (a), and then using $\backslash$ to solve yields 





