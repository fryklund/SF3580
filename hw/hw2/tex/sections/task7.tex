
%!TeX spellcheck = en-US
%!TEX root = ../hw2_report.tex
\subsection*{(a)}


Given that $A = V^{-1}\Lambda V$ we want to show $A^{k} = A = V^{-1}\Lambda^{k} V$, which is done by induction. The initial stage is

\begin{equation}
  A^{2} = V^{-1}\Lambda V V^{-1}\Lambda V =V^{-1}\Lambda^{2} V.
\end{equation}

Assume $A^{k} = V^{-1}\Lambda^{k} V$ for some nonzero $k$, then
\begin{equation}
  A^{k+1} = (V^{-1}\Lambda V)^{k}(V^{-1}\Lambda V) = V^{-1}\Lambda^{k} V\,V^{-1}\Lambda V=V^{-1}\Lambda^{k+1} V.
\end{equation}

Thus $A^{k} = V^{-1}\Lambda^{k} V$. A simple consequence is that for $p\in P^{0}_{n}$ one has

\begin{equation}
  p(A) = \sum\limits_{k = 1}^{n} a_{k}A^{k} = \sum\limits_{k = 1}^{n} a_{k}V^{-1}\Lambda^{k+1} V= V^{-1}\left(\sum\limits_{k = 1}^{n} a_{k}\Lambda^{k+1}\right) V = V^{-1}p\left(\Lambda^{k+1}\right) V
\end{equation}
with $a_{0} = 1$ for $p\in P_{n}^{0}$ and knowing that $A^{0} = I$. We have

\begin{equation}
  \min\limits_{p\in P_{n}^{0}}\|p(A)\|\leq\|V\|\|V^{-1}\| \min\limits_{p\in P_{n}^{0}}\|p(\Lambda)\|
\end{equation}
as a consequence of norms being submultiplicative.
% ########################################################################################################################
\subsection*{(b)}
First we show by induction that

\begin{equation}
  \label{eq:task7bprop}
  \begin{pmatrix}
    \lambda_{1} & 1\\
    0 & \lambda_{1}
  \end{pmatrix}^{k}
  =
  \begin{pmatrix}
    \lambda_{1}^{k} & k \lambda_{1}^{k-1}\\
    0 & \lambda_{1}^{k}
  \end{pmatrix}.
\end{equation}
The initial stage is for $k = 2$:
\begin{equation}
\begin{pmatrix}
  \lambda_{1} & 1\\
  0 & \lambda_{1}
\end{pmatrix}^{2}
=
\begin{pmatrix}
  \lambda_{1}^{2} & 2 \lambda_{1}\\
  0 & \lambda_{1}^{2}
\end{pmatrix}.
\end{equation}
Assume \eqref{eq:task7bprop} holds for $k$, then
\begin{equation}
  \begin{pmatrix}
    \lambda_{1} & 1\\
    0 & \lambda_{1}
  \end{pmatrix}^{k+1}
  =
  \begin{pmatrix}
    \lambda_{1}^{k} & k \lambda_{1}^{k-1}\\
    0 & \lambda_{1}^{k}
  \end{pmatrix}
  \begin{pmatrix}
    \lambda_{1} & 1\\
    0 & \lambda_{1}
  \end{pmatrix}
=
  \begin{pmatrix}
    \lambda_{1}^{k+1} & (k+1) \lambda_{1}^{k}\\
    0 & \lambda_{1}^{k+1}
  \end{pmatrix},
\end{equation}
i.e. the proposition \eqref{eq:task7bprop}  holds for all nonzero $k$.


Introduce the monomial $p_{k}(z) = z^{k}$, then

\begin{equation}
p_{k}\left(\begin{pmatrix}
  \lambda_{1} & 1\\
  0 & \lambda_{1}
\end{pmatrix}\right) = \begin{pmatrix}
  \lambda_{1}^{k} & k \lambda_{1}^{k-1}\\
  0 & \lambda_{1}^{k}
\end{pmatrix} = \begin{pmatrix}
  p_{k}(\lambda_{1}) &  p_{k}^{\prime}(\lambda_{1})\\
  0 & p_{k}(\lambda_{1})
\end{pmatrix}
\end{equation}
which holds for all nonzero $k$ from the induction proof above. We now have

\begin{equation}
  p\left(  \begin{pmatrix}
    \lambda_{1} & 1\\
    0 & \lambda_{1}
  \end{pmatrix}  \right) = \sum\limits_{k = 1}^{n} a_{k} p_{k}\left(\begin{pmatrix}
    \lambda_{1} & 1\\
    0 & \lambda_{1}
  \end{pmatrix}\right) = \sum\limits_{k = 1}^{n}\begin{pmatrix}
     a_{k}p_{k}(\lambda_{1}) &   a_{k}p_{k}^{\prime}(\lambda_{1})\\
    0 &  a_{k}p_{k}(\lambda_{1})\end{pmatrix} =
    \begin{pmatrix}
       p(\lambda_{1}) &   p^{\prime}(\lambda_{1})\\
      0 &  p(\lambda_{1})\end{pmatrix}.
\end{equation}
% ########################################################################################################################
\subsection*{(c)}
Let $A$ be a block diagonal matrix, such that

\begin{equation}
  A = \begin{pmatrix}
    A_{1}& & & &\\
    & A_{2} & & &\\
    & & \ddots& &\\
    & & & A_{m}&
\end{pmatrix}
\end{equation}
where $A_{i}$ are Jordan block matrices. Due to the block structure we have
\begin{equation}
  p(A) = \begin{pmatrix}
    p(A_{1})& & & &\\
    & p(A_{2}) & & &\\
    & & \ddots& &\\
    & & & p(A_{m})&
\end{pmatrix}.
\end{equation}
Each block $p(A_{i})$ has a  singular value decomposition $p(A_{i}) = U_{i}S_{i} V_{H}^{\ast}$, where $U_{i}$ and $V_{i}$ are unitary matrices. $S_{i}$ is a diagonal matrix with the singular values $\sigma$ as elements. We can now write $p(A)$ as follows.

\begin{equation}
  p(A) = \underbrace{\begin{pmatrix}
    U_{1}& & & &\\
    & U_{2} & & &\\
    & & \ddots& &\\
    & & & U_{m}&
\end{pmatrix}}_{U:=}
\underbrace{
\begin{pmatrix}
  S_{1}& & & &\\
  & S_{2} & & &\\
  & & \ddots& &\\
  & & & S_{m}&
\end{pmatrix}}_{S:=}
\underbrace{
\begin{pmatrix}
  V_{1}^{H}& & & &\\
  & V_{2}^{H} & & &\\
  & & \ddots& &\\
  & & & V_{m}^{H}&
\end{pmatrix}}_{V^{H}:=}
\end{equation}
due to the rules of multiplication for block diagonal matrices.
The final result follows from the definition of the operator norm $\|\cdot\|_{2}$:
\begin{align*}
  \|p(A)\|_{2} &= \sigma_{\max}(p(A)) = \max S = \max\limits_{i = 1,\ldots,m} \left( \max S_{i})\right)\\
&=\max\limits_{i = 1,\ldots,m} \left( \sigma_{\max}(p(A_{i}))\right)=\max\limits_{i = 1,\ldots,m} \left(\|p(A_{i})\|_{2}\right)\\
&= \max \left(\left\|\begin{pmatrix}
   p(\lambda_{1}) &   p^{\prime}(\lambda_{1})\\
  0 &  p(\lambda_{1})\end{pmatrix} \right\|_{2},|p(\lambda_{3})|,\ldots,|p(\lambda_{m})|\right)
\end{align*}
% ########################################################################################################################
\subsection*{(d)}
It is clear that
\begin{equation}
  p(z) = (\alpha_{n} + \beta_{n}z)\frac{(c-z)^{n-1}}{c^{n-1}}
\end{equation}
satisfies $p\in P_{n}$. This immediately gives $\alpha_{n} = 1$. We now study

\begin{equation}
  p^{\prime}(z) = \frac{c\left(1-\frac{z}{c}\right)^{n}(\alpha_{n} - \alpha_{n} n + \beta_{n}(c-nz))}{(c-z)^{2}}= \frac{c\left(1-\frac{z}{c}\right)^{n}(1 -  n + \beta_{n}(c-nz))}{(c-z)^{2}}.
\end{equation}
Thus
\begin{equation}
  p^{\prime}(\lambda_{1}) = 0 \Leftrightarrow \frac{c\left(1-\frac{\lambda_{1}}{c}\right)^{n}(1 -  n + \beta_{n}(c-n\lambda_{1}))}{(c-\lambda_{1})^{2}} = 0 \Leftrightarrow (1 -  n + \beta_{n}(c-n\lambda_{1})) = 0,
\end{equation}
that is

\begin{equation}
  \beta_{n} =  \frac{n-1}{c-n\lambda_{1}}.
\end{equation}
In turn this assumes that $c \neq n \lambda_{1}$ for $n > 1$.
% ########################################################################################################################

\subsection*{(e)}
Assuming $x_{n}$ is the $n$:th iterate generated by gmres, we have by lemma $2.1.3$ from that lecture notes that
\begin{align*}
  &\|Ax_{n}-b\|_{2} = \min\limits_{x \in \mathcal{K}{n}(A,b)}\|Ax - b\|_{2} = \min\limits_{p\in P_{n}^{0}}\|p(A)b\| \leq \|V\|\|V^{-1}\| \min\limits_{p\in P_{n}^{0}}\|p(\Lambda)\| \|b\|\\
  \Leftrightarrow& \frac{\|Ax_{n}-b\|_{2}}{\|b\|}\leq \|V\|\|V^{-1}\| \min\limits_{p\in P_{n}^{0}}\|p(\Lambda)\|
\end{align*}
due to the result in $7$ $(a)$. Let
\begin{equation}
q(z) = \left(1 + z \frac{n-1}{c-n\lambda_{1}}\right)\frac{(c-z)^{n-1}}{c^{n-1}}.
\end{equation}
from the previous task, which by construction is an element of $P_{n}^{0}$. Thus
\begin{equation}
  \min\limits_{p\in P_{n}^{0}}\|p(\Lambda)\| \leq \|q(\Lambda)\| =  \max \left(\left\|\begin{pmatrix}
     q(\lambda_{1}) &   q^{\prime}(\lambda_{1})\\
    0 &  q(\lambda_{1})\end{pmatrix} \right\|_{2},|q(\lambda_{3})|,\ldots,|q(\lambda_{m})|\right).
\end{equation}
Recall that $q^{\prime}(\lambda_{1}) = 0$ and that the matrix $2$-norm of diagonal matrix is the largest element in modulus. The expression above can be simplified as

\begin{equation}
  \label{eq:7f}
  \min\limits_{p\in P_{n}^{0}}\|p(\Lambda)\|\leq \max \left(|q(\lambda_{1})|,|q(\lambda_{3})|,\ldots,|q(\lambda_{m})|\right) = \max\limits_{\lambda_{i}} \left(1 + \lambda_{i} \frac{n-1}{c-n\lambda_{1}}\right)\frac{(c-\lambda_{i})^{n-1}}{c^{n-1}}.
\end{equation}

It is given that all eigenvalues are contained in the disc centered at $c$ with radius $\rho$. By taking the modulus the inequality   \eqref{eq:7f} and assuming $\lambda_{1}\neq0$ we get
\begin{align*}
\min\limits_{p\in P_{n}^{0}}\|p(\Lambda)\|\leq \max\limits_{\lambda_{i}} \left|1 + \lambda_{i} \frac{n-1}{c-n\lambda_{1}}\right|\frac{\rho^{n-1}}{|c^{n-1}|} \leq \max\limits_{\lambda_{i}}\frac{\overbrace{|c-\lambda_{i}|}^{\leq \rho} + n\overbrace{|\lambda_{i}-\lambda_{1}|}^{\leq 2\rho}}{|c-n\lambda_{1}|} \frac{\rho^{n-1}}{|c^{n-1}|}\leq \gamma_{n} \frac{\rho^{n}}{|c^{n}|}
\end{align*}
with
\begin{equation}
  \gamma_{n} =\frac{\frac{1}{n} + 2}{\left||\frac{1}{n}|-|\frac{\lambda_{1}}{c}|\right|}.
\end{equation}
We already claimed that $c\neq n\lambda_{1}$, thus the denominator is nonzero for all $n$. In the limit we have

\begin{equation}
  \lim\limits_{n\rightarrow \infty} \gamma_{n} = 2\frac{|c|}{|\lambda_{1}|}
\end{equation}
which is bounded. Combining all the results above gives

\begin{equation*}
  \frac{\|Ax_{n}-b\|_{2}}{\|b\|}\leq \|V\|\|V^{-1}\| \gamma_{n} \frac{\rho^{n}}{|c^{n}|}.
\end{equation*}

If $\lambda_{1} = 0$ then $\beta_n = (n-1)/c$ and the corresponding bound for \eqref{eq:7f} is
\begin{align*}
\min\limits_{p\in P_{n}^{0}}\|p(\Lambda)\|\leq \max\limits_{\lambda_{i}} \left|1 + \lambda_{i} \frac{n-1}{c}\right|\frac{\rho^{n-1}}{|c^{n-1}|} \leq \max\limits_{\lambda_{i}}\left(\overbrace{|c-\lambda_{i}|}^{\leq \rho} + n|\lambda_{i}|\right) \frac{\rho^{n-1}}{|c^{n}|} \leq \gamma_{n} \frac{\rho^{n}}{|c^{n}|}.
\end{align*}
However, now
\begin{equation}
\gamma_{n}  = \max\limits_{\lambda_{i}}\left(1 + n\frac{|\lambda_{i}|}{\rho}\right),
\end{equation}
which is not a bounded sequence.
% ########################################################################################################################
\subsection*{(f)}
For nonzero $\lambda_{1}$ we have convergence, but the speed is influenced by $\gamma_{n}$. Roughly, the further the centre $c$ is from $\lambda_{1}$ the better. For many iterations we approximately get
\begin{equation}
  \gamma_{n}\frac{\rho^{n}}{|c^{n}|} \approx \frac{2\rho}{|\lambda_{1}|}\frac{\rho^{n-1}}{|c^{n-1}|}.
\end{equation}
Thus the rate of convergence is the same, but the factor $\frac{2\rho}{|\lambda_{1}|}$ may be large. So if the double eigenvalues lie close to zero and the other  eigenvalues lie far away from the origin, then the factor will be large.

For $\lambda_{1} = 0$ the sequence $\gamma_{n}$ is not bounded. Note that this does not mean that GMRES will diverge, only that the estimate gives no information.
% ########################################################################################################################
\subsection*{(e)}
We discussed with Aku Kammonen and Parikshit Upadhyaya.
